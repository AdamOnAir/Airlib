\documentclass{article}

\usepackage[utf8]{inputenc}	%ces deux premiers packages
\usepackage[T1]{fontenc}		%concernent le formattage
\usepackage[french]{babel}	%Langue Français
\usepackage{hyperref}
\usepackage{listings}

\lstdefinestyle{customc}{
  belowcaptionskip=1\baselineskip,
  breaklines=true,
  frame=L,
  xleftmargin=\parindent,
  language=bash,
  showstringspaces=false,
  basicstyle=\footnotesize\ttfamily,
  keywordstyle=\bfseries\color{green!40!black},
  commentstyle=\itshape\color{purple!40!black},
  identifierstyle=\color{blue},
  stringstyle=\color{orange},
}

\lstdefinestyle{customasm}{
  belowcaptionskip=1\baselineskip,
  frame=L,
  xleftmargin=\parindent,
  language=[x86masm]Assembler,
  basicstyle=\footnotesize\ttfamily,
  commentstyle=\itshape\color{purple!40!black},
}

\hypersetup{
    colorlinks=true,
    linkcolor=blue,
    filecolor=magenta,      
    urlcolor=cyan,
    pdftitle={Overleaf Example},
    pdfpagemode=FullScreen,
}

\usepackage{mathtools, amsfonts, amsthm}
							%pour écrire des maths
\usepackage{geometry}[a4paper,textwidth=160mm,textheight=250mm] 		
%pour gérer les marges et les espaces
\usepackage{xcolor}			%pour afficher des parties en couleur

\usepackage{hyperref}		%pour effectuer des liens internes et externes au document
\usepackage{graphicx}

\numberwithin{equation}{section}    % numérotation des eqns selon la section

% Titre

\title{Documentation du Stellar Engine}
\author{Adam Ellouze}

\begin{document}				%Début du corps du document

\maketitle

\section{Airlib}
    Le Airlib est une librairie ayant une API comme \href{httŝ://flixel.org}{Flixel} et \href{https://www.openfl.org/}{OpenFL}.\vspace{1cm}

\section{Langage}
La librairie utilisera le langage C pour des raisons d'optimisations, de performances et de simplicité.

\section{Utilisation}
    Le développeur tiers - qui utilisera la librairie - devra inclure \textit{\textbf{old.h}}  et/ou \textit{\textbf{graphics.h}}, constituant la mojorité l'API.

\section{Compilation}
    Pour compiler la librairie, il faut avoir CMake installé. Voici les étapes de compilation : \vspace{0.5cm}

    \begin{lstlisting}
        git clone https://github.com/FBDev64/Airlib.git
        cd Airlib
        mkdir build && cd build
        cmake .. && make
    \end{lstlisting}
    \vspace{0.5cm}

\section{License}
    \begin{lstlisting}
        Copyright (c) 2024 Adam Ellouze <elzadam11@gmail.com>

This software is provided 'as-is', without any express or implied
warranty. In no event will the authors be held liable for any damages
arising from the use of this software.

Permission is granted to anyone to use this software for any purpose,
including commercial applications, and to alter it and redistribute it
freely, subject to the following restrictions:

1. The origin of this software must not be misrepresented; you must not
   claim that you wrote the original software. If you use this software
   in a product, an acknowledgment in the product documentation would be
   appreciated but is not required.
2. Altered source versions must be plainly marked as such, and must not be
   misrepresented as being the original software.
3. This notice may not be removed or altered from any source distribution.
    \end{lstlisting}

\end{document}				      %Fin  
